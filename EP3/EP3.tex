\documentclass[a4paper,11pt]{article}
\usepackage[T1]{fontenc}
\usepackage[utf8]{inputenc}
\usepackage{graphicx}
\usepackage{amsmath}
\usepackage{amsfonts}
\usepackage[brazilian]{babel}
\usepackage[left=2.5cm,right=2.5cm,top=2.0cm,bottom=1.5cm]{geometry}
\usepackage[hidelinks]{hyperref}
\usepackage{indentfirst}
\usepackage{caption}
\usepackage{subcaption}
\usepackage{algorithm}
\usepackage{algpseudocode}

\date{}
\author{Lucas Magno \\ 7994983}
\title{Exercício Programa 3 \\ Matrizes Ortogonais e o Problema de Quadrados Mínimos}

\begin{document}
    \maketitle

    \section*{Introdução}
    Este EP consiste em se resolver um sistema linear sobredeterminado na forma
    $$ Ax = b $$
    onde
    \begin{align*}
        A & \in \mathbb{R}^{n\times m}\\
        x & \in \mathbb{R}^{m} \\
        b & \in \mathbb{R}^{n}
    \end{align*}

    a fim de minimizar a norma do resíduo, ou, equivalentemente, sua norma ao quadrado, dada por
    \begin{align*}
        \| r \|_2^2 &= \| b - Ax \|_2^2 \\
                             &= \sum_{i = 1}^n (b_i - (Ax)_i)^2 \\
                             &= \sum_{i = 1}^n r_i^2
    \end{align*}

    Ou seja, o problema se resume em encontrar $x$ que minimize os $r_i^2$, o que dá o nome ao Método dos Mínimos Quadrados.

    \section*{Matrizes Ortogonais}
    Para tal, vale a pena o estudo de matrizes ortogonais, que são definidas como qualquer matriz $Q$ tal que
    $$ Q^TQ = I$$
    ($Q^T$ também é ortogonal).

    E portanto, como se verifica facilmente para qualquer vetor ou matriz $x$
    $$ \| Qx \|_2 = \| x \|_2$$

    Ou seja, podemos utilizar matrizes ortogonais para transformar o sistema na forma
    $$ Q^TAx = Q^Tb$$

    cujo resíduo
    $$ r = Q^Tb - Q^TAx$$

    tem a mesma norma que o sistema original
    $$ \|r\|_2 = \|Q^T(b - Ax)\|_2 = \|b - Ax\|_2$$

    \newpage
    \section*{Refletores de Householder}
\end{document}
