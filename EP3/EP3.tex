\documentclass[a4paper,11pt]{article}
\usepackage[T1]{fontenc}
\usepackage[utf8]{inputenc}
\usepackage{graphicx}
\usepackage{amsmath}
\usepackage{amsfonts}
\usepackage[brazilian]{babel}
\usepackage[left=2.5cm,right=2.5cm,top=2.0cm,bottom=1.5cm]{geometry}
\usepackage[hidelinks]{hyperref}
\usepackage{indentfirst}
\usepackage{caption}
\usepackage{subcaption}
\usepackage{algorithm}
\usepackage{algpseudocode}

\date{}
\author{Lucas Magno \\ 7994983}
\title{Exercício Programa 3 \\ Matrizes Ortogonais e o Problema de Quadrados Mínimos}

\begin{document}
    \maketitle

    \section*{Introdução}
    Este EP consiste em se resolver um sistema linear sobredeterminado na forma
    $$ \mathbf{Ax} = \mathbf{b} $$
    onde
    \begin{align*}
        \mathbf{A} & \in \mathbb{R}^{n\times m}\\
        \mathbf{x} & \in \mathbb{R}^{m} \\
        \mathbf{b} & \in \mathbb{R}^{n}
    \end{align*}

    a fim de minimizar a norma do resíduo, ou, equivalentemente, sua norma ao quadrado, dada por
    \begin{align*}
        \| \mathbf{r} \|_2^2 &= \| \mathbf{b} - \mathbf{Ax} \|_2^2 \\
                             &= \sum_{i = 1}^n (\mathbf{b}_i - \mathbf{(Ax)}_i)^2 \\
                             &= \sum_{i = 1}^n r_i^2
    \end{align*}

    Ou seja, o problema se resume em encontrar $\mathbf{x}$ que minimize os $r_i^2$, o que dá o nome ao Método dos Mínimos Quadrados.

    \section*{Matrizes Ortogonais}

\end{document}
