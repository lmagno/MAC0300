\documentclass[a4paper,11pt]{article}
\usepackage[T1]{fontenc}
\usepackage[utf8]{inputenc}
\usepackage{graphicx}
\usepackage{amsmath}
\usepackage{amsfonts}
\usepackage[brazilian]{babel}
\usepackage[left=2.5cm,right=2.5cm,top=2.0cm,bottom=1.5cm]{geometry}
\usepackage[hidelinks]{hyperref}
\usepackage{indentfirst}
\usepackage{caption}
\usepackage{subcaption}
\usepackage{algorithm}
\usepackage{algpseudocode}

\date{}
\author{Lucas Magno \\ 7994983}
\title{Exercício Programa 3 \\ Matrizes Ortogonais e o Problema de Quadrados Mínimos}

\begin{document}
    \maketitle

    \section*{Introdução}
        Este EP consiste em se resolver um sistema linear sobredeterminado na forma
        $$ Ax = b $$
        onde
        \begin{align*}
            A & \in \mathbb{R}^{n\times m}\\
            x & \in \mathbb{R}^{m} \\
            b & \in \mathbb{R}^{n}
        \end{align*}

        a fim de minimizar a norma do resíduo, ou, equivalentemente, sua norma ao quadrado, dada por
        \begin{align*}
            \| r \|_2^2 &= \| b - Ax \|_2^2 \\
                                 &= \sum_{i = 1}^n (b_i - (Ax)_i)^2 \\
                                 &= \sum_{i = 1}^n r_i^2
        \end{align*}

        Ou seja, o problema se resume em encontrar $x$ que minimize os $r_i^2$, o que dá o nome ao Método dos Mínimos Quadrados.

    \section*{Matrizes Ortogonais}
        Para tal, vale a pena o estudo de matrizes ortogonais, que são definidas como qualquer matriz $Q$ tal que
        $$ Q^TQ = I$$
        ($Q^T$ também é ortogonal).

        E portanto, como se verifica facilmente para qualquer vetor ou matriz $x$
        $$ \| Qx \|_2 = \| x \|_2$$

        Ou seja, podemos utilizar matrizes ortogonais para transformar o sistema na forma
        $$ Q^TAx = Q^Tb$$

        cujo resíduo
        $$ r = Q^Tb - Q^TAx$$

        tem a mesma norma que o sistema original
        $$ \|r\|_2 = \|Q^T(b - Ax)\|_2 = \|b - Ax\|_2$$

        Ou seja, podemos transformar o sistema em um mais simples de se resolver, com a garantia de que as soluções são a mesma.
    \newpage
    \section*{Refletores de Householder}

        Uma transformação ortogonal interessante para o nosso caso é uma reflexão de Householder, que permite a seguinte transformação de um determinado
        vetor $x \in \mathbb{R}^n$
        $$ Q\begin{bmatrix}
                x_1 \\
                x_2 \\
                \vdots \\
                x_n
            \end{bmatrix}
            \rightarrow
            \begin{bmatrix}
                -\tau  \\
                0      \\
                \vdots \\
                0
            \end{bmatrix}$$
        onde $Q$ é o refletor e $\tau = \pm \|x\|_2$.
        Melhor ainda, sua construção é bem simples, dada por
        $$ Q = I - \gamma u u^T $$
        com
        $$
            \gamma = \frac{x_1 + \tau}{\tau}\text{ , }
            u =
            \begin{bmatrix}
                1 \\
                x_2/(x_1+\tau) \\
                \vdots         \\
                x_n/(x_1+\tau)
            \end{bmatrix}
        $$
        então só $\gamma$ e $u$ precisam ser armazenados, e não a $Q$ inteira.\footnote{Também não é difícil verificar que $Q$ é simétrica e $Q = Q^T$.}

    \section*{Decomposição QR}
        Portanto, escolhendo a transformação de forma a realizar isso na primeira coluna da matriz A, com norma $\tau_1$ e a chamando de $Q_1$, temos
        $$ Q_1
        \begin{bmatrix}
            a_{11} & a_{12} & \dots & a_{1m} \\
            a_{21} & a_{22} & \dots & a_{2m} \\
            \vdots & \vdots & \ddots & \vdots \\
            a_{n1} & a_{n2} & \dots & a_{nm} \\
        \end{bmatrix}
        \rightarrow
        \left[
        \begin{array}{c|ccc}
            -\tau_1  & a_{12}^* & \dots & a_{1m}^* \\ \hline
                  0 &  &  &  \\
            \vdots &  & A_2 &  \\
                  0 &  &  &  \\
        \end{array}\right] $$

        Podemos ainda escolher uma nova transformação $\hat{Q}_2$ para realizar o mesmo na primeira coluna de $A_2$, mas precisamos compô-la com a identidade para que ela aja somente em $A_2$, não alterando o restante de $A$.
        Então
        $$
        Q_2 =
        \left[
        \begin{array}{c|ccc}
            I_1 & 0 & \dots & 0 \\ \hline
            0 & &  & \\
            \vdots & & \hat{Q}_2 & \\
            0 & &  & \\
        \end{array} \right]
        $$
        aí teremos
        $$Q_2Q_1
        \begin{bmatrix}
            a_{11} & a_{12} & \dots & a_{1m} \\
            a_{21} & a_{22} & \dots & a_{2m} \\
            \vdots & \vdots & \ddots & \vdots \\
            a_{n1} & a_{n2} & \dots & a_{nm} \\
        \end{bmatrix}
        \rightarrow
        \left[
        \begin{array}{cc|ccc}
            -\tau_1  & a_{12}^* & a_{13}^* & \dots & a_{1m}^* \\
                  0  & -\tau_2  & a_{23}^* & \dots & a_{2m}^* \\ \hline
                  0  &       0  &          &       &          \\
            \vdots   &  \vdots  &          &  A_3  &          \\
                  0  &       0  &          &       &          \\
        \end{array}\right] $$

        Escolhendo uma reflexão para cada coluna de $A$, poderíamos fazer
        $$ Q_m\dots Q_1A =
        \begin{bmatrix}
            R \\
            0
        \end{bmatrix}
        $$
        com $R \in \mathbb{R}^{m\times m}$ triangular superior.
        Na prática, no entanto, as colunas de $A$ nem sempre são linearmente independentes, e eventualmente alguma subcoluna terá norma nula (portanto $\tau = 0$), tornando a $R$ singular, o que não permitiria seu uso para resolver o sistema.

        Para evitar isso, organizar as colunas de $A$ de forma a manter as subcolunas de maiores normas à esquerda, isto é, antes de escolher cada $Q_k$ ($k = 1\dots m$), escolher a subcoluna de maior norma de $A_k$ e trocá-la com a primeira, de forma que os $\tau_k$ fiquem organizados em ordem decrescente.

        Assim, se no passo $r+1$ a maior norma das subcolunas de $A_{r+1}$ é zero, então todos os elementos dessa submatriz são nulos e temos
        $$ Q_r\dots Q_1AP = R =
        \begin{bmatrix}
            R_{11} & R_{22} \\
                 0 &      0
        \end{bmatrix}$$

        onde $R_{11} \in \mathbb{R}^{r\times r}$ é triangular superior e não-singular.\footnote{Alterar as colunas de $A$ durante os passos é equivalente a ter multiplicado ela inicialmente por uma matriz de permutação $P$.}

        Definindo
        $$ Q^T = Q_r \dots Q_1 $$
        temos
        $$ AP = QR $$
        que é a decomposição QR da matriz $A$ com pivotamento.

    \section*{Solução do Problema de Mínimos Quadrados}
        Voltando ao sistema original,
        \begin{align*}
            APP^{-1}x &= b \\
            QRP^{-1}x &= b \\
             RP^{-1}x &= Q^Tb \\
        \end{align*}

        E a resolução do sistema se resume a
        \begin{align*}
            c        &= Q^Tb \\
            R\hat{x} &= c \\
            x        &= P\hat{x}
        \end{align*}

        Sabendo a forma de $R$, no entanto, podemos calcular o resíduo \footnote{$\hat{x}_1, \hat{c} \in \mathbb{R}^{r}$}
        \begin{align*}
            r &=
            \begin{bmatrix}
                R_{11} & R_{22} \\
                     0 &      0
            \end{bmatrix}
            \begin{bmatrix}
                \hat{x}_1 \\
                \hat{x}_2
            \end{bmatrix}
            -
            \begin{bmatrix}
                \hat{c} \\
                d
            \end{bmatrix} \\
            &=
            \begin{bmatrix}
                R_{11}\hat{x}_1 + R_{22}\hat{x}_2 - c \\
                d
            \end{bmatrix}
        \end{align*}
        cuja norma ao quadrado é
        $$ \|r\|_2^2 = \|R_{11}\hat{x}_1 + R_{22}\hat{x}_2 - \hat{c}\|_2^2 + \|d\|_2^2$$

        que é quem queremos minimizar. Fica claro que não podemos alterar o termo $\|d\|_2^2$, mas podemos escolher $\hat{x}_1$ e $\hat{x}_2$ de forma a zerar o primeiro termo. Ou seja

        $$ R_{11}\hat{x}_1 = \hat{c} - R_{22}\hat{x}_2$$
        o que tem infinitas soluções, dependendo da escolha dos termos. Como é arbitrário, podemos pegar simplesmente $\hat{x}_2 = 0$ e o segundo passo da solução ($R\hat{x} = c$) se reduz resolver o sistema triangular superior não-singular
        $$ R_{11}\hat{x}_1 = \hat{c} $$
        e o resíduo se torna
        $$ \|r\|_2 = \|d\|_2 $$

    \section*{O programa}
        Esses métodos foram então implementados em um programa em C
\end{document}
