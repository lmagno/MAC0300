\documentclass[a4paper,11pt]{article}
\usepackage[T1]{fontenc}
\usepackage[utf8]{inputenc}
\usepackage{graphicx}
\usepackage{amsmath}
\usepackage[brazilian]{babel}
\usepackage[left=2.5cm,right=2.5cm,top=2.0cm,bottom=1.5cm]{geometry}
\usepackage[hidelinks]{hyperref}
\usepackage{indentfirst}
\usepackage{caption}
\usepackage{subcaption}
\usepackage{algorithm}
\usepackage{algpseudocode}

\date{}
\author{Lucas Magno \\ 7994983}
\title{Exercício Programa 2 \\ Métodos iterativos para sistemas lineares: Gradientes Conjugados}

\begin{document}
    \maketitle

    \section*{Introdução}
    Este EP consiste em implementar a resolução de sistemas lineares na forma

    $$ \mathbf{Ax} = \mathbf{b} $$

    onde
    \begin{align*}
        \mathbf{A} \in \mathbf{R}^{n\times n} & \text{, simétrica, definida positiva e esparsa} \\
        \mathbf{x}, \mathbf{b}\in \mathbf{R}^{n} & \text{, densos}
    \end{align*}
    através do método de Gradientes Conjugados.

    \section*{Motivação}
    Por que utilizar matrizes esparsas?
    A necessidade de matrizes esparsas fica mais clara quando se lida com matrizes gigantes,
    de forma que armanezar a matriz toda na memória não é uma possibilidade. No entanto,
    em diversas aplicações (diferenças finitas para resolução de equações diferenciais, por exemplo)
    surgem matrizes bem estruturadas, com apenas alguns elementos não-nulos. Assim, guardando
    apenas estes elementos, se torna possível manipular matrizes ordens de grandezas maiores
    do que caberia na memória de um computador.

    Por outro lado, embora economizem muita memória dependendo da densidade de elementos
    não nulos, geralmente o acesso aos elementos de uma matriz esparsa é muito mais caro do que aos
    de uma densa (que é $O$(1)), já que estes são armazenados utilizando estruturas que não
    permitem o acesso direto a um elemento (como uma lista ligada). Este acesso pode ser
    feito barato, a custo da eficiência de se iterar sobre as linhas ou colunas da matriz,
    sendo difícil conciliar o desempenho dessas duas operações.

    Por esses motivos, os métodos padrões de resolução de sistemas lineares (LU, Cholesky)
    não são eficazes para matrizes esparsas, pois além de acessarem diretamente elementos
    da matriz, também podem estragar sua esparsidade ao decompô-la em outras matrizes.

    É aí que entra o método de Gradientes Conjugados, uma vez que ele depende somente do
    produto matriz-vetor entre a matriz esparsa e vetores densos, o que pode ser feito
    eficaz facilmente, escolhendo uma estrutura adequada. Além disso, como não altera
    a matriz original, nem a decompõe em novas, não altera sua esparsidade, mantendo
    a eficiência de seu uso.

    \section*{Matrizes Esparsas}
    Há diversas implementações comuns de matrizes esparsas, com suas vantagens e desvantagens,
    mas aqui vamos focar nas duas utilizadas neste EP.

    \subsection*{Coordinate List (COO)}
    É uma forma simples de matriz esparsa, composta por uma cadeia ordenada de tuplas na forma
    (linha, coluna, valor). Neste trabalho foi implementada utilizando uma lista ligada,
    tornando simples a inserção de elementos novos mantendo a ordenação escolhida (por colunas
    e então por linhas).
    Entretanto, obter uma coluna específica de uma matriz neste formato não é muito eficiente,
    então ele foi utilizado somente para construção da matriz esparsa e então traduzido
    para um formato mais adequado.

    \subsection*{Compressed Sparse Column (CSC)}
    Mais complexo do que o formato COO, o CSC é composto principalmente por três vetores

    \begin{itemize}
        \item \textbf{nzval}: contém os valores de todos os elementos não-nulos, ordenados por
        coluna e então por linha
        \item \textbf{rowval}: contém os valores dos índices de linha de cada elemento correspondente em \texttt{nzval}
        \item \textbf{colptr}: contém os índices de \texttt{nzval} em que se encontram os primeiros elementos de cada coluna da matriz. Por exemplo, \texttt{nzval(colptr(j))} representa
        o valor do primeiro elemento da coluna \texttt{j} na matriz.
    \end{itemize}
    Vale notar que \texttt{nzval} e \texttt{rowval} são obtidos diretamente a partir de uma matriz no formato COO, sendo necessário apenas o cálculo de \texttt{colptr}.
    A vantagem de se utilizar este formato está na facilidade de se obter uma coluna da matriz (a coluna \texttt{j} são os elementos em \texttt{nzval} entre \texttt{colptr(j)} e \texttt{colptr(j+1)-1}), o que torna barato o produto matriz-vetor orientado a colunas, mais especificamente, da ordem do número do elementos não nulos da matriz, já que basta iterar sobre \texttt{nzval}.

    \section*{Gradientes Conjugados}
    Como $\mathbf{A} \in \mathbf{R}^{n\times n}$ é uma matriz definida positiva, vale que

    $$ \mathbf{x^{T}Ax} \geq \mathbf{0} $$

    onde a igualdade vale se e somente se $\mathbf{x} = \mathbf{0}$ .
    Assim, $\mathbf{A}$ define uma norma e portanto um produto interno.
    Podemos então definir que dois vetores $\mathbf{u}$ e $\mathbf{v}$ são ortogonais se e somente se

    $$ \mathbf{u^{T}Av} = \mathbf{0}$$

    Tendo isso, podemos pegar um conjunto $P$ de $n$ vetores $\mathbf{p_i}$ ortogonais entre si, que então formam uma base do $\mathbf{R}^n$ e podemos escrever qualquer vetor $ \mathbf{x} \in \mathbf{R}^n$ como uma combinação linear dos $\mathbf{p_i}$

    $$ \mathbf{x} = \sum_i \alpha_i \mathbf{p_i} $$

    onde os $\alpha_i$ são as projeções de $\mathbf{x}$ em $\mathbf{p_i}$:

    $$ \alpha_i = \frac{\mathbf{p_i Ax}}{\mathbf{p_i^TAp_i}} = \frac{\mathbf{p_i b}}{\mathbf{p_i^TAp_i}} $$

    Então bastaria escolher os $\mathbf{p_i}$ e calcular as projeções. No entanto, mesmo isso pode ser custoso, pois estamos lidando com $n$ muito grande.
    Na verdade, podemos escolher os $\mathbf{p_i}$ de forma que apenas alguns deles sejam necessários para uma boa aproximação de $\mathbf{x}$, então é possível construir um algoritmo iterativo, que se aproxima da solução verdadeira e a alcançaria em $n$ passos, se fosse utilizada precisão infitita, uma vez que isso representa a soma de todas as projeções, que é a identidade.

    Definindo $\mathbf{x_\star}$ como a solução de $\mathbf{Ax_\star} = \mathbf{b}$, queremos então minimizar o erro definido por
    $$ \mathbf{e} = \mathbf{x_\star} - \mathbf{x}$$

    mas não temos como calcular $\mathbf{e}$, pois não possuímos $\mathbf{x_\star}$. Entretanto, vale que
    $$ \| \mathbf{e} \|_A \propto \frac{1}{2}\mathbf{x^TAx - x^Tb} $$

    Como queremos minimizar essa função, poderíamos escolher as projeções (que podem ser vistas como direções de passo) no sentido contrário de seu gradiente (que é $\mathbf{Ax - b}$), mas também queremos manter os $\mathbf{p_i}$ ortogonais entre si. Logo, se definirmos $\mathbf{r_i} = \mathbf{b - Ax_i}$ (o resíduo no passo $i$), podemos definir
    $$ \mathbf{p_i} = \mathbf{r_i} - \sum_{k<i} \mathbf{\frac{p_k^T A r_i}{p_k^T A p_k} p_k}$$

    Explorando as ortogonalidades entre $\mathbf{p}$, $\mathbf{r}$ e $\mathbf{x}$, pode-se mostrar que

    \begin{align*}
        \mathbf{x_{i+1}} &= \mathbf{x_i} + \alpha_i\mathbf{p_i}  \\
        \mathbf{r_{i+1}} &= \mathbf{r_i} - \alpha_i\mathbf{Ap_i} \\
        \mathbf{p_{i+1}} &= \mathbf{r_i} + \beta_i\mathbf{p_i}
    \end{align*}

    onde

    \begin{align*}
        \alpha_i &= \mathbf{\frac{r_i^Tr_i}{p_i^TAp_i}} \\
        \beta_i  &= \mathbf{\frac{r_{i+1}^Tr_{i+1}}{r_i^Tr_i}}
    \end{align*}

    Portanto, o algoritmo fica
    \begin{algorithm}
        \caption{Gradientes Conjugados}
        \begin{algorithmic}[1]
            \State $\mathbf{r_0} \gets \mathbf{b - Ax_0}$
            \State $\mathbf{p_0} \gets \mathbf{r_0}$
            \State $i \gets 0$
            \Loop
                \State $\alpha_i \gets \frac{\mathbf{r_i}\cdot\mathbf{r_i}}{\mathbf{p_i\cdot Ap_i}}$
                \State $\mathbf{x_{i+1}} \gets \mathbf{x_i} + \alpha_i\mathbf{p_i}$
                \State $\mathbf{r_{i+1}} \gets \mathbf{r_i} - \alpha_i\mathbf{Ap_i}$

                \If {$\sqrt{\mathbf{r_{i+1}}\cdot\mathbf{r_{i+1}}} < \epsilon$}

                    \textbf{exit}
                \EndIf

                \State $\beta_i \gets \frac{\mathbf{r_{i+1}}\cdot\mathbf{r_{i+1}}}{\mathbf{r_i}\cdot\mathbf{r_i}}$
                \State $\mathbf{p_{i+1}} \gets \mathbf{r_{i+1}} + \beta_i\mathbf{p_i}$
                \State $k \gets k + 1$
            \EndLoop

        \Return $\mathbf{x_{i+1}}$

        \end{algorithmic}
    \end{algorithm}

    em que a matriz $\mathbf{A}$ só aparece em um produto matriz-vetor.

    \section*{O Programa}
        Para testar o algoritmo, foram geradas matrizes esparsas aleatórias utilizando o método descrito em [1], variando-se o tamanho das matrizes
        e sua densidade desejada $\tau$.
        Como esse método não garante que a matriz gerada seja definida positiva, foram utilizados os módulos do EP1 (Cholesky e resolução de sistema tridiagonal)
        para verificar isso, além de resolver novamente o sistema a fim de comparação.

    \section*{Resultados}
        Os sistemas foram resolvidos para matrizes com lado $n = 2, 4, \dots, 4096$ e
        densidades $\tau = 0.01, 0.05, 0.1, 0.2$.

        Os tempos de execução e dados no geral foram impressos para a tela enquanto que as
        normas do resíduos em cada passo da solução por Gradientes Conjugados foram impressas
        para arquivos externos para serem plotados.

    \begin{thebibliography}
        \bibitem{bau97}
            Lloyd N. Trefethen \and David Bau,
            \emph{Numerical Linear Algebra},
            Society for Industrial and Applied Mathematics,
            1997.
    \end{thebibliography}
\end{document}
